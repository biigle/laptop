%%%%%%%%%%%%%%%%%%%%%%%%%%%%%%%%%%%%%%%%%%%%%%%%%%%%%%%%%%%%
% Preamble:
%%%%%%%%%%%%%%%%%%%%%%%%%%%%%%%%%%%%%%%%%%%%%%%%%%%%%%%%%%%%

% KOMA article class
% see also:
% http://mirror.ctan.org/macros/latex/contrib/koma-script/scrguide.pdf
\documentclass[english]{scrartcl}

\usepackage[utf8]{inputenc}
\usepackage[T1]{fontenc}
\usepackage[final]{microtype}
\usepackage{fancyhdr}
\pagestyle{fancy}

% use more space for the text (a4)
\renewcommand{\footskip}{1cm}
\renewcommand{\textheight}{690pt}
\renewcommand{\textwidth}{400pt}

% latin modern - 'The fonts, as compared to the CM family,
% contain a lot of additional characters, mainly accented
% ones.' http://www.ctan.org/tex-archive/fonts/lm/
\usepackage{lmodern}

% Linemargin
\usepackage{setspace}
\onehalfspacing

\usepackage[english]{babel}
% \usepackage[ngerman]{babel}

% e.g.: \includegraphics ...
\usepackage{graphicx}
\usepackage{subcaption}
\usepackage{float}

% e.g. \align ...
\usepackage{amsmath}
% prettier fonts
% \usepackage{mathpazo}

% nicer tables:
% \usepackage{colortbl}
\usepackage{booktabs}

% code
\usepackage{listings}
\usepackage{color}

\lstset{
	%backgroundcolor=\color[rgb]{0.95, 0.95, 0.95},
	xleftmargin=20pt,
	tabsize=2,
	rulecolor=,
	basicstyle=\footnotesize\ttfamily,
	aboveskip={1\baselineskip},
	columns=fixed,
	showstringspaces=false,
	extendedchars=true,
	breaklines=true,
	prebreak = \raisebox{0ex}[0ex][0ex]{\ensuremath{\hookleftarrow}},
	frame=l,
	numbers=left,
	numberstyle=\scriptsize\ttfamily,
	showtabs=false,
	showspaces=false,
	showstringspaces=false,
	identifierstyle=\ttfamily,
	keywordstyle=\color[rgb]{1.0,0,0},
	keywordstyle=[1]\color[rgb]{0,0,0.75},
	keywordstyle=[2]\color[rgb]{1,0.0,0.5},
	keywordstyle=[3]\color[rgb]{0.127,0.427,0.514},
	keywordstyle=[4]\color[rgb]{0.4,0.4,0.4},
	commentstyle=\color[rgb]{0.5,0.5,0.5},%\color[rgb]{0.133,0.545,0.133},
	stringstyle=\color[rgb]{0.639,0.082,0.082},
}

% special characters in biblio
\usepackage{bibgerm}

% advanced tables
\usepackage{bigdelim}
\usepackage{multirow}
\usepackage{longtable}

% rotate
\usepackage{rotating}
\def\rot#1{\begin{sideways}#1\end{sideways}}

%paragraph formatting
\usepackage{parskip}

% always last:
% e.g. \url, \href, inter-PDF-links ...
\usepackage{hyperref}

% path for graphics
\graphicspath{{./img/}}

% defs
